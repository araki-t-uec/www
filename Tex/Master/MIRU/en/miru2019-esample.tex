% MIRU2019: How to use the LaTeX class Ver.1
%
\documentclass[MIRU,submit,english]{miru2019e}
\usepackage{graphicx}
%\usepackage{latexsym}
%\usepackage[fleqn]{amsmath}
%\usepackage[psamsfonts]{amssymb}

\begin{document}

\title{How to Use \LaTeXe\ Class File for MIRU2019}

\affiliate{Tokyo}{First University}
\affiliate{Osaka}{Second University (Presently with Third Corporation)}

 \author{Hanako GAZO}{Tokyo}[hanako@gazo.ac.jp]
 \author{Taro NINSHIKI}{Osaka}[taro@ninshiki.co.jp]
 \author{Jiro RIKAI}{Osaka}[jiro@rikai.co.jp]

%Abstract and keywords should be omitted

\maketitle

\section*{Abstract}
Please write abstract about 200 words.

\section{Introduction}
This is a sample for MIRU2019 papers.
The paper is compiled using ASCII Japanese p\LaTeXe. 

\section{How to prepare extended abstract}

\subsection{Language}

Both Japanese and English are acceptable. 

\subsection{Paper length}

Papers are limited to four pages. 

\subsection{Author name and affiliation}

Please list full names, affiliations and email addresses of authors.
Do not provide mailing (physical) addresses. 
The process of selecting oral papers is single blind.

\subsection{keywords}

keywords must be omitted.

%\subsection{Miscellaneous}
%
%Do not change the paper format, including font size, baseline skip, paper margins. The change may aff%ect the result of paper acceptance.

%\bibliographystyle{miru2019e}
%\bibliography{myref}

\begin{thebibliography}{9}% If the number of articles and books is more than 9, use {99} instead of {9}.
\bibitem{1}
W. Rice, A. C. Wine, and B. D. Grain,
diffusion of impurities during epitaxy,
Proc. IEEE, vol.~52, no.~3, pp.~284--290, March 1964.
\bibitem{2}
 H. Tong, Nonlinear Time Series: A Dynamical System Approach, J. B. Elsner, ed., Oxford University Press, Oxford, 1990.
\bibitem{3}
H. K. Hartline, A. B. Smith, and F. Ratlliff,
Inhibitoryinteraction in the retina,
in Handbook of Sensory Physiology,
ed. M. G. F. Fuortes, pp.~381--390, Springer-Verlag, Berlin.
\bibitem{4}
Y. Yamamoto, S. Machida, and K. Igeta,
``Micro-cavity semiconductors with enhanced spontaneous emission,''
Proc. 16th European Conf. on Opt. Commun.,
no.~MoF4.6, pp.~3--13, Amsterdam, The Netherlands, Sept.1990.
\end{thebibliography}

\end{document}
