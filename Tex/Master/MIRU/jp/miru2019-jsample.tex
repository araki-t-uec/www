% MIRU2019 LaTeXクラスファイルの使い方 Ver.1
\documentclass[MIRU,submit]{miru2019j}
\usepackage{graphicx}
%\usepackage{latexsym}
%\usepackage[fleqn]{amsmath}
%\usepackage[psamsfonts]{amssymb}

\begin{document}

\title{MIRU2019 \LaTeXe\ クラスファイルの使い方}

\affiliate{Tokyo}{第一大学}
\affiliate{Osaka}{第二大学(現在,第三コーポレーション勤務)}

 \author{画像 花子}{Hanako GAZO}{Tokyo}[hanako@gazo.ac.jp]
 \author{認識 太郎}{Taro NINSHIKI}{Osaka}[taro@ninshiki.co.jp]
 \author{理解 次郎}{Jiro RIKAI}{Osaka}[jiro@rikai.co.jp]

% 概要・キーワードは省略して下さい.

\maketitle

\section*{概要}
本稿は,MIRUの2019用のサンプルです.今年より200文字程度の概要を記述してください.

\section{はじめに}
本稿はMIRU2019用の原稿サンプルです.
アスキー版 p\LaTeXe\ に基づいて作成しています.

\section{原稿の書き方}

\subsection{言語}
使用言語は,日本語または英語です.

\subsection{ページ数}
ページ数は4ページ以下として下さい.

\subsection{著者名と所属}
著者名と所属,連絡先メールアドレスを原稿に明記して下さい.住所の記載は不要です.
口頭発表候補論文の評価方法はシングルブラインドとなります.

\subsection{キーワード}
キーワードは省略して下さい.

%\bibliographystyle{miru2019j}
%\bibliography{myref}

\begin{thebibliography}{9}% 文献数が10未満の時 {9}
\bibitem{1}
W. Rice, A. C. Wine, and B. D. Grain,
diffusion of impurities during epitaxy,
Proc. IEEE, vol.~52, no.~3, pp.~284--290, March 1964.
\bibitem{2}
 H. Tong, Nonlinear Time Series: A Dynamical System Approach, J. B. Elsner, ed., Oxford University Press, Oxford, 1990.
\bibitem{3}
H. K. Hartline, A. B. Smith, and F. Ratlliff,
Inhibitoryinteraction in the retina,
in Handbook of Sensory Physiology,
ed. M. G. F. Fuortes, pp.~381--390, Springer-Verlag, Berlin.
\bibitem{4}
Y. Yamamoto, S. Machida, and K. Igeta,
``Micro-cavity semiconductors with enhanced spontaneous emission,''
Proc. 16th European Conf. on Opt. Commun.,
no.~MoF4.6, pp.~3--13, Amsterdam, The Netherlands, Sept.1990.
\end{thebibliography}

\end{document}
