\chapter{おわりに}
\section{まとめ}
本研究ではテレビ映像中から特定の動作を認識し検出した. 顔認識, 食事画像認識, 食事動作認識を組み合わせることで精度の向上を図った. 
顔認識は人間が写っていないショットを取り除くために使用した. 
食事画像認識ではfoodcnnを用いることで食事, 非食事クラスでの分類率を90.0 \%とすることができた. 
動作認識ではSVMによる5-fold cross validationによる評価を用いて92.5 \%となった. 
最終的にweb上で検出結果を表示した. それぞれの認識手法を組み合わせることで, 誤検出されるショットの数を大幅に減少させることができ, また平均適合率は最もよい結果を示すことができた. 
しかし, 今回の結果は食事シーンを認識するための十分な精度は得られなかった. 


\section{今後の課題}
現在の認識手法では動作検出の精度を向上させなければ他の認識手法を組み合わせても認識できる食事シーンの数が変わらない. 原因として考えられるのは今回の学習用データにはテレビ映像のデータが入っていないことである. 
しかし本システムを利用することで, 動作のショットを簡単に集めることができるようになった. 今後は本システムで集めたテレビ映像のデータを学習用データとして加えていくことで全体の精度の向上を図ることができる. \par
また今回は実験として「食べる」動作に対してのみに絞って検出をした. 今後は別の動作も認識, 検出できるように拡張させる必要がある. 
顔認識, 物体認識, 動作認識を組み合わせて実験を行ったが, 今回は姿勢の検出や, 字幕情報などを利用していない. 
このような他の手法を加えていくことでさらなる精度向上を目指す. 


\newpage