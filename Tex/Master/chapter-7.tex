\chapter{考察}
\chapter{まとめ,今後の課題}
動画の各フレームの平均を取り,画像として識別した.
データの少ないクラスは精度が低いため,データを補う必要がある.
予備実験では簡易的な方法を用いたが,今後は最新手法による分類を検討している.またレスキュー犬の行動を認識する際には複数クラスの出力にする必要がある.


今後の課題として,時系列情報を特徴量抽出に使う.
また,音声データから特徴量を抽出し,動画特徴量と併せたマルチモーダルな特徴量を利用し,レスキュー犬の行動分類を行う.


課題
リアルタイムにやりたい
音声フレームの長さが適しているかわからない
%{\scriptsize % 7pt
%{\footnotesize % 8pt
%{\small % 9pt
%\bibliographystyle{ieee}
\bibliographystyle{junsrt}
\bibliography{ref}
%}
% \begin{footnotesize}
% %{\small
% \bibliography{ref}
% \bibliographystyle{junsrt}
% %}
% \end{footnotesize}
\end{document}

