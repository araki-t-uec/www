\chapter{考察}
\begin{table}[tb]
 \centering
 \caption{各実験比較表}\label{expetiments_result}
 \scalebox{0.90}[1.00]{
  \begin{tabular}{|l||c|c|c|c|c|c|c|c|c|c|c|c|}
   \hline \hline
   & \rotatebox{90}{bark}& \rotatebox{90}{cling}&\rotatebox{90}{command}& \rotatebox{90}{eat}&\rotatebox{90}{handler}& \rotatebox{90}{run}&\rotatebox{90}{victim}& \rotatebox{90}{shake}& \rotatebox{90}{sniff}& \rotatebox{90}{stop}& \rotatebox{90}{walk} & \rotatebox{90}{全体}\\ \hline
静止画像   & 0.244& 0.066& 0.0& 0.024& 0.057& 0.0& 0.204& 0.0& 0.0& 0.588& 0.51&  0.436 \\ \hline
optical flow   & 0.141& 0.0& 0.0& 0.0& 0.017& 0.0& 0.017& 0.0& 0.0& 0.586& 0.476&  0.406 \\ \hline
音声 (Conv1D)   & 0.669& 0.078& 0.22& 0.023& 0.138& 0.0& 0.274& 0.44& 0.502& 0.745& 0.704&  0.512 \\ \hline
音声 (Conv2D)   & 0.563& 0.04& 0.188& 0.001& 0.059& 0.0& 0.201& 0.304& 0.524& 0.744& 0.74&  0.512 \\ \hline
静止画像+optical flow   & 0.11& 0.018& 0.043& 0.0& 0.155& 0.0& 0.259& 0.0& 0.426& 0.705& 0.668&  0.435 \\ \hline
静止画像+音声   & 0.662& 0.031& 0.195& 0.018& 0.115& 0.002& 0.308& 0.402& 0.498& 0.726& 0.694&  0.5 \\ \hline
optical flow+音声   & 0.667& 0.054& 0.234& 0.014& 0.123& 0.01& 0.223& 0.356& 0.487& 0.759& 0.692&  0.493 \\ \hline
静止画像+optical+音声   & 0.577& 0.135& 0.186& 0.066& 0.183& 0.026& 0.433& 0.409& 0.53& 0.779& 0.725&  0.518 \\ \hline
  \end{tabular}
 }
\end{table}




\chapter{まとめ,今後の課題}
Sound based Three-streamの提案と,提案手法を用いたレスキュー犬の行動推定を行なった.
提案手法との比較のために行なった実験は以下である.
\begin{itemize}
  \item 静止画像からの犬行動マルチラベル推定
  \item optical flow画像からの犬行動マルチラベル推定
  \item 音声からの犬行動マルチラベル推定
  \item 静止画像とoptical flow画像からの犬行動マルチラベル推定
  \item 音声と静止画像からの犬行動マルチラベル推定
  \item 音声とoptical flow画像からの犬行動マルチラベル推定
  \item 音声と静止画とoptical flow画像からの犬行動マルチラベル推定
\end{itemize}
結果は提案手法が最も高く,音声・静止画像・optical flow画像のそれぞれに必要な情報が含まれており,3つのデータにそれぞれ必要な情報が含まれているとわかった.
精度は51.2\%と数値では決して高いとは言えないが,約30fpsの動画で1/5フレーム毎に行なった推定の結果であるため非実用的とも言えない結果となった.

今回は研究の範囲としなかったが,レスキュー犬行動動画の入力に対してリアルタイムに結果を出すことも求められる.
また,実験に対する疑問として音声フレームの長さはどの程度か適しているのか判断できていない.
これの変更でどの程度影響があるのかを検証し,より適切な音声フレーム長が判断できればより高い精度も期待できるのではないだろうか.

%{\scriptsize % 7pt
%{\footnotesize % 8pt
%{\small % 9pt
%\bibliographystyle{ieee}
\bibliographystyle{junsrt}
\bibliography{ref}
%}
% \begin{footnotesize}
% %{\small
% \bibliography{ref}
% \bibliographystyle{junsrt}
% %}
% \end{footnotesize}
\end{document}

