\documentclass[twocolumn, a4paper]{UECIEresume}

\usepackage[dvipdfmx]{graphicx}
\usepackage{graphicx}
\usepackage{amsmath}
\usepackage{txfonts}

\title{レスキュー犬の一人称動画を用いた動作分類}
\date{平成 30 年 9 月 27 日}
\affiliation{総合情報学科 メディア情報学コース}
\supervisor{柳井啓二 教授,ZZZ 准教授}
\studentid{1730010}
\author{荒木勇人}
%\headtitle{平成 yy 年度 総合情報学科 卒業論文中間発表}
%\headtitle{平成 yy 年度 総合情報学科 卒業論文発表}
\headtitle{平成 30 年度 総合情報学科 修士論文中間発表}
%\headtitle{平成 yy 年度 総合情報学科 修士論文発表}

\begin{document}
\maketitle

\section{はじめに}
 被災地での救助活動を行う際に,人間だけでは不十分な場合がある.そこで,人間に代わって探査を行うのが訓練されたレスキュー犬(災害救助犬)である.レスキュー犬は,がれきの隙間などの狭い空間,倒壊した建物など踏破困難な環境でも探査可能であり,また人間より発達した嗅覚を頼りにした救助活動が可能である.しかし,彼らは人間に向けた言語を持たないため,人間はレスキュー犬の行動から彼らが収集した情報を理解しなくてはならない.現状では,災害救助犬を指揮するハンドラーと呼ばれる人間がレスキュー犬の行動を手動でマーキングしており,その情報を消防などの指揮命令者に口頭伝達している.この問題点として,トリアージ(緊急度に従った手当の優先順位付け)のための周辺環境情報や,要救助者の情報量の不足や客観性の不足があげられる.
 レスキュー犬の行動をモニタリングするために,装着型計測・記録装置が開発された~\cite{dog01}.これは,各種センサを用いた計測データを記録し,リアルタイムに映像などのデータを無線配信している.本研究では,これらのデータを用いてレスキュー犬の行動をリアルタイムに分類すること目的とする.深層学習を用いた動画識別にある既存手法を予備実験として行った.動画識別のための新規性としては,映像だけでなく音声などのデータを使ったマルチモーダルな画像分類という点があげられる.
レスキュー犬が今何をしているのかが明示的に判明することで,トリアージに必要な情報が整理され,災害救助活動の効率化が期待される.

\section{フォーマット}
はじめにの導入の次に実験手法やモデルについての説明を行います.
これには図などを交えて記述することが望ましいでしょう.

また適切な所で段落を切って下さい.段落はロジックのひとまとまりを表すもので,
読者にどう読ませるかを指示するものです.これが不適切だと文書は読みにくくなります.

文書のフォーマットは,
\begin{itemize}
  \item{サイズは A4.卒論は 1 ないし 2 ページ,修論は 2 ページ}
  \item{2段組を原則とし,おおよそ1行24文字,1ページ45行程度}
  \item{マージンは上下 25mm, 左右 20mm}
  \item{タイトルは,表題,発表者(コース名,学籍番号,氏名),指導教員を明記すること}
  \item{図や表に関しても上記をはみ出さないようにしてください}
  \item{日本語フォントはタイトルがゴシック 12pt, 本文が 9〜10 pt程度とします}
  \item{ヘッダは表ページ,左側に ``平成 xx 年度卒業論文中間発表'' のような,右側は学籍番号を入れることとします.
      裏ページに関しては何も設定しなくても構いません.
    }
  \item{各セクションの表題はゴシックを用いて判るようにすること}
\end{itemize}
とします.このサンプルファイルに合うような体裁で記述して下さい.
\LaTeX を用いる場合は,添付クラスファイルを用いると楽です.
(日本語の場合は jsarticle.cls が必要です.)

数式を文中に入れる場合は,必ず数式モードを使うようにして下さい.
MS-Word などを使う場合は,数式エディタを用いて記述して下さい.


\section{結果の提示など}
実験結果などは,わかりやすく,図や表を使ってまとめて下さい.予稿は殆どスペースがないので,
これ1枚あれば説明できるという図表を貼り付けると効果的です.
また図表を貼りつけた場合は,キャプションを入れるとともに,
かならず本文中でも説明を行なって下さい.



\section{まとめ}
最後は簡潔に研究成果をまとめて下さい.将来の課題などもあれば書いても良いですが,あまり課題を書きすぎると逆効果になりますのでほどほどにしておきましょう.

また,引用文献はキチンと入れましょう\cite{Kinoshita1}.引用は,先人に対するリスペクトなので,よほど独立性が高い研究でない限り必要となります.
\begin{footnotesize}
%{\small
\bibliographystyle{junsrt}
\bibliography{ref}
%}
\end
\end{document}
